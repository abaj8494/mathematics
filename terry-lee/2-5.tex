\documentclass{article}

\usepackage{enumitem}
\usepackage{geometry}
\usepackage{multicol}
\usepackage{amsmath}
\usepackage{accents}
\geometry{top=20mm,bottom=20mm,right=20mm,left=20mm}

\title{Terry Lee Vectors Review}
\author{Aayush Bajaj}
\date{}

\begin{document}
\maketitle

\dotfill

\begin{multicols}{2}
\section*{1}
\begin{enumerate}
    \item[a)] \(\vec{AC}\)
    \item[b)] \(\vec{AC}\)
    \item[c)] \(2\vec{AB}\)
    \item[f)] \(3\vec{BA}\)
\end{enumerate}

\section*{2}

\begin{enumerate}
    \item[a)] \(\vec{AB} = (-4, 2)\text{, and }\vec{BC} = (14, -7)\text{. Then since } \vec{BC} = 3.5\vec{AB}\text{, the points A, B and C are collinear}\)
    \item[b)] \(k = 6\)
    \item[d)] \(m + 4 = -6 \Rightarrow m = -2. n - 5 = 2 \Rightarrow n = 7.\)
\end{enumerate}
\end{multicols}

\begin{multicols}{2}
\section*{5}
Parallel = \(b\)\\
Orthogonal = \(b, c\)

\section*{6}
\begin{enumerate}
    \item[a)] \(|\undertilde{a}| = \sqrt{13}. |\undertilde{b}| = \sqrt{5}. \theta = 2.09 rad\)
    \item[d)] \(|\undertilde{a}| = \sqrt{17}. |\undertilde{b}| = \sqrt{35}. \theta = 1.61 rad\)
\end{enumerate}

\end{multicols}

\begin{multicols}{2}
\section*{7}
\begin{enumerate}
    \item[a)] \((2,3)\)
    \item[b)] \(2,-3\)
\end{enumerate}

\section*{8}
\(\theta = 1.86 rad\)

\end{multicols}

\begin{multicols}{2}
\section*{9}
    \begin{enumerate}
        \item[a)] \(\undertilde{a} = 2\undertilde{b} - \undertilde{c}\). So, \(\undertilde{a}\) is linearly dependent on \(\undertilde{b} \text{ and } \undertilde{c}\) and therefore these vectors span a plane. 
        \item[b)] 
    \end{enumerate}

\section*{10}
\end{multicols}





\end{document}
